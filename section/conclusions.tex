\chapter{Hasil Review}

Tak ada gading yang tak retak. Panduan ini juga masih banyak kekurangan, dan perlu banyak jam terbang untuk evaluasi. Pada bagian ini contoh bagaimana sebuah paper mengalami review dan penolakan dari paper yang dikirim kepada jurnal Q1, Q2 dan Q3. 

\section{Artikel Ditolak Q1}
Beberapa contoh artikel ditolak dengan isi komentar :
\begin{enumerate}
	\item We are writing to inform you that we will not be able to process your paper further. Papers sent for peer-review are selected on the basis of discipline, novelty and general significance, in addition to the usual criteria for publication in scholarly journals. Therefore, our decision is not necessarily a reflection of the quality of your research. We wish you every success if you choose to submit the paper elsewhere.
	\item The manuscript should have a structured abstract (Background/ introduction, Methods, Results and Conclusions).
	\item As part of our revised review processes, new submissions can be reviewed by a senior member of the editorial staff for a `fit/no fit' decision.  This can save great time for the authors and avoid lengthy review procedures.  A review of this manuscript has been completed and we do not believe it is a good fit for DSS or its readership.  I see no research contribution in the submission.
	\item I see no research contribution in the submission.  It is a straightforward analysis of one year of a very limited data set.
	\item Dear Mr. Awangga,

	Thank you for your interest and submission to Expert Systems With Applications (ESWA). 

	Kindly note that the  manuscripts first goes through a preliminary screen by the Editorial Office to see whether they should be passed to the next stage of formal and rigorous peer reviews.   Important screening criteria in ESWA include (1) fit with the mission of the ESWA, (2) significance, originality, and impact on expert and intelligent systems, and (3) likelihood of moving forward acceptance in two rounds of rigorous peer review process.  We are sorry that your paper didn't make the initial cut. The manuscript failed to meet these criteria, and was inappropriate for publication at this time.   

	However, I do think it could be considered by another journal, and I would like to suggest that you take advantage of the article transfer service that `Expert Systems With Applications' is part of. This gives you the option to have your manuscript files and details transferred to another journal. This removes the need for you to resubmit and reformat your manuscript, saving you valuable time and effort during the submission process.

	If you click the link below you will find relevant information about the journal(s) to which I recommend transferring your submission. You have the option to accept or decline the transfer offer from the same web page:

	This offer does not constitute a guarantee that your paper will be published in the suggested Journal, but it is our hope that this arrangement will help expedite the process for promising papers.

	To learn more about the new article transfer service, please visit

	With kind regards,
	\item Dear Mr. Awangga,

	Thank you for submitting your manuscript to Heliyon. Unfortunately, after reviewing your paper the editor believes that it is not suitable for publication in the journal and is unlikely to be favorably reviewed by the referees. Accordingly, the manuscript is being returned without review.

	Thank you for giving us the opportunity to consider your work.

	Kind regards,

	Miss Elizabeth Wetherell
	Editorial Assistant
	Heliyon

	Editor's Comments:

	The manuscript does not meet the quality standards of a manuscript submitted to a respected journal. It looks more like a textbook exercise rather than a research article. Moreover, the readability of the manuscript is an indication that the authors did not prepare meticulously the paper. 

	Does this study report original research and conclusions?

	There is no original research in the manuscript. This is just a rather illustrative example of two methods that are used in any quantitative methods 101 class.

	Are the methods appropriate and described in sufficient detail?

	Two well-known methods (k-means and moving average) are applied without any novelties or contributions. Their application resembles a textbook exercise.

	Has the data been analyzed with appropriate and clearly defined statistical tests?

	Not really. There aren't any stated hypotheses, any formal statistical procedures, or another data analysis methodology.

	Are the conclusions a reasonable extension of the results?

	It's hard to accept that the results are a compelling resolution of the (in any case ill-defined) problem. Actually, the conclusions provided are the numerical results of the two exercises and do not match the quality of an expected conclusions section in a respected journal.

\end{enumerate}

\section{artikel Ditolak Q3}
Berikut adalah komentar reviewer pada artikel yang ditolak di Q3:
\begin{enumerate}
	\item There is no new idea in the proposed system. The English very poor.The authors should state the contribution of the paper internationally. In addition, requires some native speaker to fix the writing.
	\item Paper has been written on `National Border Agency Communication Behaviour Clustering Using Centrality and Meanshift'. 

	In this paper authors are analysis of boarder security and communication between other countries.What is a contribution in this paper. Not given methodology and algorithm of system. Need to more explanation of results.
	\item The text reports to the problem that often occurs in the Riau Island Province, such as border issues, illegal fishing, drug smuggling, potential transit routes of international terrorism, hazardous waste disposal and human trade, as well as underlying social issues such as health, education, housing and implementation of Asian economic community. As a solution, the author proposes the treatment of the data to generate information
	*********************
	The author did not pay attention to the technical part of the writing, presenting errors as:
	2.1. Eigenvector Centrality 2.2. Eigenvector Centrality
	Equation 1.1.
	**********************
	Errors in writing: detection [14], the certain would be, detection [14].
	**********************
	The text is very weak, with poor results and consequently a weak conclusion. The author does not compare with other algorithms.
	
\end{enumerate}