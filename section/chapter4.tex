\chapter{Pertemuan 4}

\section{Issues \#31}
Pada \textit{issues \#31} (\textit{Google Maps Feature}) \textit{chatbot} akan menggunakan fitur \textit{google maps} sehingga algoritmanya yaitu user mengirim location lalu mengetikkan kata kunci "gmaps (tujuan)" sehingga chatbot akan mengirimkan link yang akan menunjukkan jalan ketempat yang ingin dituju.

\section{Issues \#32}
Pada \textit{issues \#32} (\textit{Use Try and Except to Maps}) dengan menggunakan metode \textit{Try and Except} untuk mencari kelas yang berisi lokasi, jika ada, fungsi gmaps akan dijalankan.

\section{Issues \#33}
Pada \textit{issues \#33} (\textit{gmaps.click}) fungsi ini digunakan untuk mengklik \textit{maps} yang user kirim kepada chatbot.

\section{Issues \#34}
Pada \textit{issues \#34} (\textit{Switch BrowserTab}) setelah fungsi \textit{gmaps.click()} dijalankan maka akan membuka tab baru dan mengharuskan untuk menggunakan metode \textit{switch\_window} untuk mengganti tab pada browser.

\section{Issues \#35}
Pada \textit{issues \#35} (\textit{searchboxinput}) di\textit{field "searchboxinput"} ada value lokasi koordinat lokasi user, kita harus ambil koordinat tersebut untuk kita tampung di variabel koordinat yang fungsinya untuk \textit{direction}.

\section{Issues \#36}
Pada \textit{issues \#36} (\textit{taparea Class Name}) sebuah \textit{class name} pada \textit{HTML} yang dapat digunakan pada selenium untuk diproses seperti klik.

\section{Issues \#37}
Pada \textit{issues \#37} (\textit{tactile\_searchbox\_input}) sebuah nama kelas pada \textit{field} yang ada di \textit{HTML} pada halaman \textit{google maps}. Pada field ini akan diisi oleh variabel koordinat yang kita dapatkan dari nama kelas "searchboxinput".

\section{Issues \#38}
Pada \textit{issues \#38} (\textit{currenturl}) sebuah variabel yang akan diisi oleh \textit{url} setelah bot mendapatkan \textit{direction} pada proses google maps.

\section{Issues \#39}
Pada \textit{issues \#39} (\textit{destination Variable}) sebuah variabel string yang kita dapatkan dari pesan user setelah kata kunci "gmaps".

\section{Issues \#40}
Pada \textit{issues \#40} (\textit{sb\_cb50 id name}) adalah sebuah nama id yang ada di \textit{HTML} yang berfungsi untuk menghapus sebuah field pada "searchbox".

